\documentclass[UTF8]{ctexart}

\usepackage[a4paper, left=3.17cm, right=3.17cm, top=2.54cm, bottom=2.54cm]{geometry}
\usepackage[T1]{fontenc}
\usepackage{mathptmx}
\usepackage{amsmath}
\usepackage{amsfonts}
\usepackage{chemformula}
\usepackage{cite}
\usepackage[colorlinks, linkcolor=black, anchorcolor=black, citecolor=black]{hyperref}
\usepackage{graphicx}
\usepackage{fancyhdr}
\usepackage{enumitem}
\usepackage{listings}

\setlength{\parskip}{0.5em}
\title{\heiti 面向对象程序设计\ \ 大程M3报告}
\author{\Large \kaishu \textup{刘泓健\ \ 吴一航}}

\ctexset {
    section = {
        format={\Large \bfseries \heiti} 
        } 
    }

\lstset{
    columns=fixed,       
    numbers=left,                                        % 在左侧显示行号
    numberstyle=\tiny\color{gray},                       % 设定行号格式
    frame=none,                                          % 不显示背景边框
    backgroundcolor=\color[RGB]{245,245,244},            % 设定背景颜色
    keywordstyle=\color[RGB]{40,40,255},                 % 设定关键字颜色
    numberstyle=\tiny,keywordstyle=\color{blue!70},
    commentstyle=\color{red!50!green!50!blue!50},frame=shadowbox,
    rulesepcolor=\color{red!20!green!20!blue!20},basicstyle=\ttfamily,
    stringstyle=\rmfamily\slshape\color[RGB]{128,0,0},   % 设置字符串格式
    showstringspaces=false,                              % 不显示字符串中的空格
    language=C++,                                        % 设置语言
    }

\pagestyle{fancy}
\fancyhf{}
\cfoot{\thepage}
\fancyhead[C]{面向对象程序设计\ \ 大程M3报告}
\begin{document}

\begin{titlepage}
	\newcommand{\HRule}{\rule{\linewidth}{0.5mm}}
	\begin{figure}
        \flushleft
        \includegraphics[scale=0.4]{0.png}
    \end{figure}
    \center 
	\quad\\[1.5cm]
	\textsl{\Large Zhejiang University }\\[0.5cm] 
	\textsl{\large College of Computer Science and Technology}\\[0.5cm] 
	\makeatletter
	\HRule \\[0.4cm]
	{ \huge \bfseries \@title}\\[0.4cm] 
	\HRule \\[1.5cm]
	\begin{minipage}{0.4\textwidth}
		\begin{flushleft} \Large
			\emph{Author:}\\
			\@author 
		\end{flushleft}
	\end{minipage}
	~
	\begin{minipage}{0.4\textwidth}
		\begin{flushright} \Large \kaishu
			\emph{Supervisor:} \\
			\textup{翁恺}
		\end{flushright}
	\end{minipage}\\[3cm]
	\makeatother
	{\large An Assignment submitted for the ZJU:}\\[0.5cm]
	{\large {211C0010\ \ 面向对象程序设计}}\\[0.5cm]
	{\large \today}\\[2cm] 
	\vfill 
\end{titlepage}
    
    \section{概述}
    在这两周的大程设计中,我们在此前完成的基本类设计基础上,完成了
    dialog类以及函数的设计以及相关的测试程序,我们的MUD游戏初始场景已经可以
    成功运行。

    \section{代码基本框架}
    本月的大程设计实现的代码基本框架如下:
\begin{lstlisting}
.
|——assets(游戏所需图片、文案等资源)
     |——dialog.json
|——doc(大程文档,如规范性文档以及注释文档等)
     |——general.md(规范文档)
|——include(自主设计的头文件)
     |——character.h
     |——skill.h
     |——community.h
     |——infrastructure.h
     |——policy.h
     |——technology.h
|——lib(外部引用库,如JSON parser及图形界面等所需外部库)
     |——nlohmann
          |——...
|——src(大程源代码)
     |——core
          |——dialog.cpp
     |——utils
     |——main.cpp(初始执行文件)
     |——Makefile
|——testbench(测试文件)
     |——xcharacter.cpp
     |——xjson.cpp
     |——xdialog.cpp
|——Makefile
\end{lstlisting}
    
    \section{dialog设计}
    在这一部分中,我们完成了开场白的设计。当程序运行时(注意:程序的运行方式
    在根目录README中),会进入游戏初始化界面,然后开始我们的初始对话。
    主函数的循环中,我们直到没有下一条对话时结束循环,结束游戏。

    在对话类的设计中,我们关注以下几个重要的属性:
    \begin{enumerate}
        \item branch:分支,根据玩家所作出的选择进入某一分支
        \item randBranch:随机分支,玩家无需做出选择,系统自动创造随机分支
        \item nextDialog:指向下一dialog的信息
    \end{enumerate}
    
    当然后续还会有条件分支的设计,这些将会是下一阶段完善剧情时会逐步实现的内容。

    我们不难发现,在运行游戏时,主体结构是首先运行当前分支的内容,然后根据玩家选择或者系统设定获取下一分支,
    接下来构造下一分支继续执行。特别注意的是,我们是通过json来实现数据的存储的,
    我们利用上一次报告中提及的json parser,取出我们dialog.json中的数据然后进行
    类的初始化等。并且这一parser还有自带的异常处理,当用户给出错误的输入时会自动
    抛出异常。当然,这一异常处理我们会在下一次整体框架完善时做进一步的改进。
    
    \section{下一阶段展望}
    在下一阶段的设计中,我们将会完成整个游戏的设计。这一方面是会将我们准备好的
    后续文案放置在asset中,另一方面是完善community、infrastructure等类
    的设计,构建起完整的文明游戏框架。除此之外,异常处理部分我们也会对原先的
    库中的处理做一定的优化,最后会根据时间决定是否使用图形库设计。

\end{document}