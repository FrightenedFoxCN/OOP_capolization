\documentclass[UTF8]{ctexart}

\usepackage[a4paper, left=3.17cm, right=3.17cm, top=2.54cm, bottom=2.54cm]{geometry}
\usepackage[T1]{fontenc}
\usepackage{mathptmx}
\usepackage{amsmath}
\usepackage{amsfonts}
\usepackage{chemformula}
\usepackage{cite}
\usepackage[colorlinks, linkcolor=black, anchorcolor=black, citecolor=black]{hyperref}
\usepackage{graphicx}
\usepackage{fancyhdr}
\usepackage{enumitem}
\usepackage{listings}

\setlength{\parskip}{0.5em}
\title{\heiti 面向对象程序设计\ \ 大程M2报告}
\author{\Large \kaishu \textup{刘泓健\ \ 吴一航}}

\ctexset {
    section = {
        format={\Large \bfseries \heiti} 
        } 
    }

\lstset{
    columns=fixed,       
    numbers=left,                                        % 在左侧显示行号
    numberstyle=\tiny\color{gray},                       % 设定行号格式
    frame=none,                                          % 不显示背景边框
    backgroundcolor=\color[RGB]{245,245,244},            % 设定背景颜色
    keywordstyle=\color[RGB]{40,40,255},                 % 设定关键字颜色
    numberstyle=\tiny,keywordstyle=\color{blue!70},
    commentstyle=\color{red!50!green!50!blue!50},frame=shadowbox,
    rulesepcolor=\color{red!20!green!20!blue!20},basicstyle=\ttfamily,
    stringstyle=\rmfamily\slshape\color[RGB]{128,0,0},   % 设置字符串格式
    showstringspaces=false,                              % 不显示字符串中的空格
    language=C++,                                        % 设置语言
    }

\pagestyle{fancy}
\fancyhf{}
\cfoot{\thepage}
\fancyhead[C]{面向对象程序设计\ \ 大程M2报告}
\begin{document}

\begin{titlepage}
	\newcommand{\HRule}{\rule{\linewidth}{0.5mm}}
	\begin{figure}
        \flushleft
        \includegraphics[scale=0.4]{0.png}
    \end{figure}
    \center 
	\quad\\[1.5cm]
	\textsl{\Large Zhejiang University }\\[0.5cm] 
	\textsl{\large College of Computer Science and Technology}\\[0.5cm] 
	\makeatletter
	\HRule \\[0.4cm]
	{ \huge \bfseries \@title}\\[0.4cm] 
	\HRule \\[1.5cm]
	\begin{minipage}{0.4\textwidth}
		\begin{flushleft} \Large
			\emph{Author:}\\
			\@author 
		\end{flushleft}
	\end{minipage}
	~
	\begin{minipage}{0.4\textwidth}
		\begin{flushright} \Large \kaishu
			\emph{Supervisor:} \\
			\textup{翁恺}
		\end{flushright}
	\end{minipage}\\[3cm]
	\makeatother
	{\large An Assignment submitted for the ZJU:}\\[0.5cm]
	{\large {211C0010\ \ 面向对象程序设计}}\\[0.5cm]
	{\large \today}\\[2cm] 
	\vfill 
\end{titlepage}
    
    \section{概述}
    在本月的大程设计中,我们在代码方面完成了游戏中所需基本类的设计,
    并且找到了适用的json parser,使得我们能够更方便地建立本游戏的数据库,
    成功启动游戏。此外,我们的json parser以及基本类都通过了基本测试。
    其他方面,我们完成了代码规范文件,也初步完成了游戏所需文案设计。

    \section{代码基本框架}
    本月的大程设计实现的代码基本框架如下:
\begin{lstlisting}
.
|——assets(游戏所需图片、文案等资源)
|——doc(大程文档,如规范性文档以及注释文档等)
     |——general.md(规范文档)
|——include(自主设计的头文件)
     |——character.h
     |——skill.h
     |——community.h
     |——infrastructure.h
     |——policy.h
     |——technology.h
|——lib(外部引用库,如JSON parser及图形界面等所需外部库)
     |——nlohmann
          |——\dots
|——src(大程源代码)
     |——core
          |——character.cpp
     |——utils
     |——main.cpp(初始执行文件)
     |——Makefile
|——testbench(测试文件)
     |——xcharacter
     |——xjson
|——Makefile
\end{lstlisting}
    
    \section{基本类设计}
    \begin{enumerate}
        \item character类
        
        基本角色类型,角色有姓名、技能、生产力、领导力以及宗教信仰等属性,
        也有相关的关于这些属性的基本方法。
        \item skill类
        
        技能类型,包括技能号、等级等属性,也有相关的基本方法。特别要注意的是,
        其中定义了character为其友元类,原因在于我们希望将这两个类区分,但是
        character类中有大量地方需要直接引用skill类属性并且其它类中不需要这些属性,
        因此我们设计此友元类。
        \item community类
        
        社群类型。包括名字、设施(vector类型)、人口以及领导(character类型)等属性
        以及相关的基本方法。
        \item infrastructure类
        
        基础设施类型,包括id、名称、等级和生产力等属性与相关的基本方法。
        \item policy类
        
        政策类型,包括id、名称以及政策影响等属性与相关的基本方法。
        \item technology类

    \end{enumerate}

    \section{JSON parser}
    在这一部分中,我们根据StackOverflow的推荐指引,选择了GitHub上的开源项目
    nlohmann作为我们游戏的数据库的JSON parser。这一JSON parser使用较为简便,
    适合于这一游戏的设计。这一部分放置在文件中的lib文件夹中,
    并通过了适应性测试。

    \section{游戏启动设计}
        
    \section{其他设计}
    我们完成了代码规范文件,详见/doc/general.md,也初步完成了游戏所需文案设计,
    这一部分待游戏设计深入时会会有更直观地体现。
\end{document}